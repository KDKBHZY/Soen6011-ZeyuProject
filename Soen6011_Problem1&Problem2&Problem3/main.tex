\documentclass[letterpaper, 11pt]{report}
\usepackage{titlesec}
\usepackage{fullpage} % changes the margin
\usepackage{amsmath}
\usepackage{amssymb}
\usepackage{graphicx} %package to manage images
\usepackage[linkcolor=red]{hyperref}
\usepackage{paralist}
\usepackage{makecell}
\usepackage{subcaption}
\usepackage{algpseudocode}
\usepackage{algorithm}
\graphicspath{ {./images/} }
\setlength\parindent{0pt}
\begin{document}
\begin{titlepage}
\vspace*{0.7in}
\begin{center}
\begin{figure}[htb]
\begin{center}
\includegraphics[width=8cm]{univ_logo}
\end{center}
\end{figure}
\vspace*{0.3in}
\begin{Large}
\textbf{SOEN 6011 : SOFTWARE ENGINEERING PROCESSES} \\
\end{Large}
\vspace*{0.1in}
\begin{Large}
\textbf{SUMMER 2022} \\
\end{Large}
\vspace*{0.9in}
\begin{Large}
\textbf{F2: Tangent Function, $tan(x)$} \\
\end{Large}
\vspace*{0.625in}
\rule{80mm}{0.1mm}\\
\vspace*{0.1in}
\begin{large}
Author \\
\vspace*{0.1in}
Zeyu Huang\\

\vspace*{0.3in}
\date{\normalsize\today} 
\end{large}
\end{center}
\begin{center}
https://www.overleaf.com/project/62cdb1d5b13422add2cceb12\end{center}
\end{titlepage}
\tableofcontents
\newpage
\addcontentsline{toc}{section}{1) Problem 1}
\addcontentsline{toc}{subsection}{a) Description of Function}
\section*{1)Problem1}
\subsection*{a)Description of Function}
 \normalsize{ \cite{test1} $tan(x)$  is a periodic function which is very important in trigonometry. The simplest way to understand the tangent function is to use the unit circle. For a given angle measure $\theta$ draw a unit circle on the coordinate plane and draw the angle centered at the origin, with one side as the positive  x -axis. The  x -coordinate of the point where the other side of the angle intersects the circle is $cos(θ)$ and the  y -coordinate is $sin(θ)$. So, the tangent function is define as below: \[tan(x) = \frac{sin(x)}{cos(x)}\]\\}
The below graph shows values corresponding to different angles.
 \begin{center}
\includegraphics[width= 6cm]{images/tan3.png}
\end{center}
 \\
 \normalsize{ \cite{test1}\cite{varsitytutors}The tangent function is undefined when $x$ $=$ $\pi$ / 2 $+$ $n \pi$ (where, $n$ is integer) for which, $cos(x) = 0$. However, Tangent function does not have an amplitude. In addition, The graph intercept $x$-axis at $n\pi$ (where $n$ is integer) and in $y$-axis at $(0,0)$ point. The period of tangent function is $\pi$.
 }
 \\
 \subsection*{Range}\cite{test1}\cite{varsitytutors}
 \normalsize{ The range of $tan(x)$ is all real number $\mathbb{R}$, $(- \infty, + \infty)$. }
 
 \subsection*{Domain and Co-domain}\cite{test1}\cite{varsitytutors}
 \normalsize{The domain of tangent function is $x \in$ $\mathbb{R}$, $x$ $\neq$ $\pi$ / 2 $+$ $n \pi$ where, $n$ is an integer. The co-domain of $tan(x)$ is \((-\infty, +\infty\)).}

\pagebreak
\newpage
\addcontentsline{toc}{subsection}{b) Context of Use Model}
\section*{b)Context of Use Model}
\normalsize{Users can use the calculator to calculate the result of $sin()$, $cos()$ and $\frac{sin()}{cos()}$ which is $tan()$ of a degree. This degree shall be an integer or decimal, so the digits \textit{0-9} and the decimal point must be available by the user. The user can select the appropriate function they want to use, and they shall be able to press a button to have the answer computed. The calculator should return the result or an error message that indicates why it was unable to do so.}
\begin{center}
\includegraphics[width=15cm]{context_diagram}
\end{center}
\pagebreak
\newpage
\addcontentsline{toc}{section}{2) Problem 2}
\section*{2)Problem 2}
\subsection*{Assumption:} 
 For the given degree x, return the result of $tan(x)$. If the input value is invalid or cannot be calculated, return an error message.

\\
\subsection*{Requirements:} \\
\begin{tabular}{ |c|c| } 
 \hline
 \textbf{Requirement Id} & R1 \\ 
   \hline
 \textbf{Overview} & $x = 0^\circ $ + $n\pi$  \\
  \hline
  \textbf{Description} & 
\begin{tabular}[c]{@{}l@{}} For the given input $x = 0^\circ $ ,
the function \\may return 0 as output.
\end{tabular} \\
  \hline
\textbf{Priority} & High \\ 
  \hline
\textbf{Type} & Functional \\
  \hline
\textbf{Difficulty} & Easy \\
  \hline
\end{tabular}

\bigskip

\begin{tabular}{ |c|c|  } 
 \hline
 \textbf{Requirement Id} & R2 \\ 
   \hline
 \textbf{Overview} &  x is Positive Degree  \\
  \hline
  \textbf{Description} & 
\begin{tabular}[c]{@{}l@{}} For the given input x = any Positive Degree  ,\\
the function may return corresponding\\ $tan(x)$ value as output.
\end{tabular} \\
  \hline
\textbf{Priority} & High \\ 
  \hline
\textbf{Type} & Functional \\
  \hline
\textbf{Difficulty} & Medium \\
  \hline
\end{tabular}\\
\bigskip
\newline
{\leftskip=0cm\relax
\begin{tabular}{ |c|c|  } 
 \hline
 \textbf{Requirement Id} & R3 \\ 
   \hline
 \textbf{Overview} &  x is Negative  Degree  \\
  \hline
  \textbf{Description} & 
\begin{tabular}[c]{@{}l@{}} For the given input x = any Negative Degree  ,\\
the function may return corresponding\\ $tan(x)$ value as output.
\end{tabular} \\
  \hline
\textbf{Priority} & High \\ 
  \hline
\textbf{Type} & Functional \\
  \hline
\textbf{Difficulty} & Medium \\
  \hline
\end{tabular}\\
\bigskip
\newline
\begin{tabular}{ |c|c| } 
 \hline
 \textbf{Requirement Id} & R4 \\ 
   \hline
 \textbf{Overview} & $x = 90^\circ$ + $n\pi$  \\
  \hline
  \textbf{Description} & 
\begin{tabular}[c]{@{}l@{}} For the given input x ,
the function \\may return "Invalid" as output.
\end{tabular} \\
  \hline
\textbf{Priority} & High \\ 
  \hline
\textbf{Type} & Functional \\
  \hline
\textbf{Difficulty} & Hard \\
  \hline
\end{tabular}

\pagebreak
\newpage
\addcontentsline{toc}{section}{3) Problem 3}
\section*{3)Problem 3}
\addcontentsline{toc}{subsection}{a) Algorithm Selection}
\subsection*{a)Algorithm Selection}
For this part, I will introduce two algorithms for implementing $tan(x)$ function. \textbf{Polynomial approximation and Maclaurin series.}\\
\newline
\textbf{Algorithm1 : Polynomial approximation} }\\
\newline
\cite{poly}Polynomial approximation is an approximation of a curve with a polynomial. When we solve mathematical questions, we don't actually know how to calculate certain functions, such as the $sin()$ function. Therefore, to solve this kind of problems, mathematicians develop very good approximations to these functions - related functions which are very close to the function of interest, but much easier to calculate.\\
\newline
\begin{tabular}{ |c|c|}
\hline
\textbf{Advantages} & \textbf{Disadvantages}\\ \hline 
Easy to calculate
 & \makecell{The approximation is only precise for small x, so some steps are \\needed when we calculate $tan(x)$} \\
\hline

\end{tabular} \\ \\

\newline
\textbf{Algorithm2 : Maclaurin series}\\
\newline
\cite{macl}A Maclaurin series is a Taylor series expansion of a function about 0,\\
 \begin{center}
$f(x) = f(0)+f'(0)x + \frac{f''(0)}{2!}x^2 + \frac{f^{(3)}(0)}{3!}x^3 + ...  + \frac{f^{(n)}(0)}{n!}x^n$\\
\end{center}
\newline
The $tan(x)$ function's approximation is derived by the Maclaurin Series's explicit forms of $sin(x)$ and $cos(x)$.
\begin{equation} 
sin(x) = x-x^3/3!+x^5/5!-x^7/7!+.....
\end{equation}
\begin{equation} 
cos(x) = 1-x^2/2!+x^4/4!-x^6/6!+.....
\end{equation}\\
Then, we can use $\tan(x)$ = $\frac{sin(x)}{cos(x)}$ to calculate.\\
\newline
\begin{tabular}{ |c|c|}
\hline
\textbf{Advantages} & \textbf{Disadvantages}\\ \hline 
\makecell{The formula  $\tan(x)$ = $\frac{sin(x)}{cos(x)}$ \\is easy to understand.}
 & \makecell{Successive terms get very complex and hard to derive.} \\
\hline

\end{tabular} \\ \\
\pagebreak
\newpage
\addcontentsline{toc}{subsection}{b) Mind Map for Pseudocode Format}
\subsection*{b)Mind Map For Pseudocode}
In this part, I will use a mind map to decide a pseudocode format.\\

\includegraphics[width= 14cm]{images/mindmap.png}
 \\
 
\addcontentsline{toc}{subsection}{c) Pseudocode for each Algorithm}
\subsection*{c) Pseudocode for each Algorithm}
In this part, I will write pseudocode for each algorithm.\\
\begin{algorithm}
\caption{\cite{pseupoly}Polynomial approximation} \label{alg:cap}
\begin{algorithmic}
\Require $x \notin [0^\circ,180^\circ]$ 
\Function{periodicity}{x}
\If{$x>180^\circ$}
\While{$x>180^\circ$}
\State $x = x-180^\circ$\Comment{reduce x to the range $[0^\circ,180^\circ]$}
\EndWhile 
\Else
       \While{$x<0^\circ$}
\State $x = x+180^\circ$\Comment{add x to the range $[0^\circ,180^\circ]$}
\EndWhile 
\EndIf\\   
\Return{$x$}\Comment{get valid x}
\EndFunction
\end{algorithmic}
\end{algorithm}
\begin{algorithm}
\begin{algorithmic}
\Require $x \notin [0^\circ,90^\circ]$ 
\Function{symmetry}{x}
\State $tan(x)$ = -$tan(180^\circ-x)$\Comment{use the symmetry of $tan()$ }\\
\Return{$tan(x)$}\Comment{get valid x}
\EndFunction\\
\newline
\Require $x \notin [0^\circ,45^\circ]$ 
\Function{cofunction}{x}
\State $tan(x)$ = -$\frac{1}{tan(90^\circ-x)}$\Comment{use the reciprocal of $tan()$ }\\
\Return{$tan(x)$}\Comment{get valid x}
\EndFunction\\
\newline
\Require $x \notin [0^\circ,22.5^\circ]$ 
\Function{ trigonometric\_identity}{x}
\State $tan(x)$ = -$\frac{2tan(\frac{x}{2})}{1-tan^2(\frac{x}{2})}$\Comment{use the trig identity of $tan()$ }\\
\Return{$tan(x)$}\Comment{get valid x}
\EndFunction\\
\newline
\Require $x \in [0^\circ,22.5^\circ]$ 
\Function{ polynomial}{x}
\State x = $x * \frac{\pi}{180^\circ}$\Comment{convert x to radians }
\State $tan(x)$ = $x+\frac{x^3}{3} + \frac{2x^5}{15} +\frac{17x^7}{315}$\Comment{use the trig identity of $tan()$ }\\
\Return{$tan(x)$}\Comment{get valid x}
\EndFunction\\
\end{algorithmic}
\end{algorithm}
\begin{algorithm}
\caption{Maclaurin Series} \label{alg:cap}
\begin{algorithmic}
\Require x in degrees
\Function{getrad}{x}
\State $val = x * \frac{\pi}{180^\circ}$\Comment{Calculate x in radians}\\
\Return{$val$}\Comment{return x in radians}
\EndFunction
\newline
\Require $n\neq 0$ and $length = 0$ 
\Function{checkdigits}{n}
\While{$n\leqq1$}
\State $length += 1$\Comment{Calculate length}
\State $n *= 10$
\EndWhile\\
\Return{$length$}\Comment{return length}
\EndFunction
\end{algorithmic}
\end{algorithm}

\begin{algorithm}
\begin{algorithmic}
\Require $getrad(x)\neq NULL$ AND $n \neq 0$ AND x in radians AND $k = 1$ AND $m = 0$
\Function{CalculateSin}{getrad(x)}
\State $sinres = \frac{x}{k!}$
\While{$checkdigits(sinres)\neq n$}
\State $k = k+2$
\If{$m \% 2 ==0$}
\State $sinres -= \frac{x^k}{k!}$ 
\Else
\State $sinres += \frac{x^k}{k!}$ 
\EndIf
\State $m+=1$
\EndWhile\\
\Return{$sinres$}\Comment{get value of sin(x)}
\EndFunction\\
\newline
\Require $getrad(x)\neq NULL$ AND $n \neq 0$ AND x in radians AND $k = 2$ AND $m = 0$
\Function{CalculateCos}{getrad(x)}
\State $cosres = 1$
\While{$checkdigits(cosres)\neq n$}
\State $k = k+2$
\If{$m \% 2 ==0$}
\State $sinres -= \frac{x^k}{k!}$ 
\Else
\State $sinres += \frac{x^k}{k!}$ 
\EndIf
\State $m+=1$
\EndWhile\\
\Return{$cosres$}\Comment{get value of cos(x)}
\EndFunction\\

\Require $getrad(x)\neq NULL$ AND $ calculatecos(x)\neq NULL$ AND $calculatesin(x)\neq NULL$
\Function{Calculatetan}{calculatecos(x),calculatesin(x)}
\State SinVal = calculatesin(x)
\State CosVal = calculatecos(x)\\
\Return{$\frac{SinVal}{CosVal}$}\Comment{calculation for tan(x)}
\EndFunction
\State $result \gets $\Call{$tan(x)$}{}
\end{algorithmic}
\end{algorithm}

\begin{thebibliography}{}
\bibitem{test1}
Varsity Tutors.
\\https://www.varsitytutors.com/hotmath/hotmath$\_$help/topics/tangent-function

\bibitem{varsitytutors} 
varsitytutors.graphing tangent function. 
\\https://www.varsitytutors.com/hotmath/hotmath$\_$help/topics/graphing-tangent-function

\bibitem{poly} 
Polynomial approximation
\\https://www.expii.com/t/what-is-a-polynomial-approximation-317
\bibitem{macl} 
Maclaurin series
\\https://mathworld.wolfram.com/MaclaurinSeries.html
\bibitem{pseupoly} 
Pseudocode for Polynomial approximation\\https://mathonweb.com/help\_ebook/html/algorithms.htm
\end{thebibliography}
\end{document}
